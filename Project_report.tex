\documentclass[]{article}

%%%PAGE DIMENSIONS
\usepackage{geometry} % to change the page dimensions
\geometry{a4paper}

%%%GRAPHICS
\usepackage{graphicx} 

%%% PACKAGES
\usepackage{bm}
\usepackage{amsmath}
\usepackage{amssymb}
\usepackage{booktabs} % for much better looking tables
\usepackage{array} % for better arrays (eg matrices) in maths
\usepackage{paralist} % very flexible & customisable lists (eg. enumerate/itemize, etc.)
\usepackage{verbatim} % adds environment for commenting out blocks of text & for better verbatim
\usepackage{subfig} % make it possible to include more than one captioned figure/table in a single float
\usepackage{color}


%%% HEADERS & FOOTER
\usepackage{fancyhdr} % This should be set AFTER setting up the page geometry
\pagestyle{fancy} % options: empty , plain , fancy
\renewcommand{\headrulewidth}{0pt} % customise the layout...
\lhead{}\chead{}\rhead{Numerical Solution of Ordinary Differential Equations M5N7}
\lfoot{Chloe Raymont}\cfoot{\thepage}\rfoot{00733439}

%%% SECTION TITLE APPEARANCE
\usepackage{sectsty}

%%% ToC (table of contents) APPEARANCE
\usepackage[nottoc,notlof,notlot]{tocbibind} % Put the bibliography in the ToC
\usepackage[titles,subfigure]{tocloft} % Alter the style of the Table of Contents

\newcommand{\comments}[2]{{\bfseries #1 says: #2}}

%%TITLE
\title{Project Notes}
\author{Chloe Raymont}

\begin{document}

\maketitle

\begin{abstract}

\end{abstract}

\section{Notes on Optimal Transport}
	\begin{itemize}
		\item Given domain $X \subseteq \mathbb{R}^2$ and discrete set of points $Y = \left\lbrace y_i, \quad 1\leq i \leq N \right\rbrace  \subset \mathbb{R}^2$ 
		\item Source measure $\mu(A) = \int_A \rho(x)dx$, $\rho$ a probability density
		\item Target measure $\nu = \sum_{1\leq i \leq N}\nu_i \delta_{y_i}$ 
		\item Pushforward of a measure $\mu$ by a map $T: X \rightarrow Y$ is $T_{\#}\mu = \sum_{y \in Y} \mu \left( T^{-1}(y) \right) \delta_{y}$ (ie) the sum of the measures of the pre-images of the points $y \in Y$ under the mapping $T$
		\item Voronoi cells defined by $\text{Vor}_y := \left\lbrace x \in K \; \text{st} \; \forall z \in Y \; c(x,y) \leq c(x,z) \right\rbrace$
		\item Laguerre cells defined by $\text{Lag}_y(\psi) := \left\lbrace x \in K \; \text{st} \; \forall z \in Y \; c(x,y) + \psi(y) \leq c(x,z) + \psi(z) \right\rbrace$
		\item The cost function used is the normal Euclidean distance \comments{Colin}{squared} on $\mathbb{R}^2$, $c(x,y) = \| x-y\|^2$
		\item A map $T: X \rightarrow Y$ between $\mu$ and the probability measure on $Y$, $\mu$ is a Transport Map $T$ if $T_{\#}\mu = \nu$
		\item The weights represent an additional cost parameter, for example in the analogy to bakeries in Paris the $\psi_i$ represents the price of a loaf at the bakery at position $y_i$. A greater value for $\psi$ would reduce the area of the Laguerre cell associated with $y_i$.
		\item It is shown by M\'{e}rigot, Meyron and Thibert \cite{Merigot2017} (proposition 13) that the transport map given by $T_\psi: x \rightarrow \text{argmin}_i\| x - y_i \|^2 + \psi_i$, where $\psi_i = \psi(y_i)$ is a family of weights on $Y$
		
		\item The problem is then finding the weights $\psi_i$ associated to the points $y_i$ such that $G_i(\psi) := \mu (\text{Lag}_{y_i}(\psi)) = \nu_i$. The Damped Newton's Algorithm from M\'{e}rigot, Meyron and Thibert (2017) finds such $\psi_i$
		
	\end{itemize}
\section{Applying Optimal Transport Method to solve the Semi-Geostrophic Equations}
	\subsection{Description of the Problem}
	Starting from the Eady Model for baroclinic instability given in Cullen [2006]
	\begin{align}
	-fv_g + \frac{\partial \varphi}{\partial x} = 0,\\
	\frac{Dv_g}{Dt} + fu -\frac{Cg}{\theta _0}\left(z-H/2\right) = 0,\\
	\frac{D\theta'}{Dt} - Cv_g = 0,\\
	\frac{\partial \varphi}{\partial z} - g\frac{\theta'}{\theta_0} = 0,\\
	\nabla \cdot \bm{u} = 0.
	\end{align}
	with periodic boundary conditions in $x$ \color{red} $x(L) = x(-L), \; u(L) = u(-L)$, \color{black} \comments{Colin}{This doesn't make sense: $x(L)=x(-L)$, this bc just applies to all fields ($v_g,u,\theta',\phi$)}
        and rigid-lid boundary conditions in $z$ (ie) $w = 0$ on $z = 0, H$
	\begin{itemize}
		\item Cullen \cite{Cullen2008} shows that the solution of equations (1) - (5) can be reformulated as an optimal transport problem where the energy integral represents the cost to be minimised.
		\begin{equation}
		E = \int_ \Gamma f^2 \left(\frac{1}{2}\left(\left(x - \tilde{X}\right)^2 +\left(y - \tilde{Y}\right)^2\right) - z\tilde{Z}\right)dxdydz
		\end{equation}
	\end{itemize}
	\begin{itemize}
	\item Initialise a set of $N$ points in physical space $(x_i,z_i) \; i = 1\,...\,N$, on domain $\Gamma = [-L,L] \times [0,H]$
	\item Using given form of $\theta ' = N_0^2 \theta_0 z /g + B\sin(\pi(x/L +z/H))$ we find $v_g$ using equations (1) and (4) from the SG Equations. Integrate (4) in $z$ and use (1) to find 
	\begin{equation}
		v_g = \frac{BgH}{\theta_0 Lf}\sin\left( \pi \left( \frac{x}{L} + \frac{z}{H}\right)\right) - \frac{2BgH}{\pi \theta_0 Lf}\cos\left(\frac{\pi x}{L}\right)
	\end{equation}
	\item We use $v_g$ to apply the Hoskin's transformation \cite{Hoskins1975}
	\begin{equation*}
	X = f^{-1}v_g + x,\qquad \left(Y = -f^{-1}u_g + y\right), \qquad Z = \frac{g\theta'}{f^2\theta_0}
	\end{equation*}
	to find a corresponding set of points $(X_i,Z_i) \; i = 1 \,...\,N$ in Transformed \color{red} Geostrophic?\color{black} Space satisfying
	\begin{align*}
	\frac{DX}{Dt} -\frac{Cg}{f\theta _0}\left(z-H/2\right) = 0,\\
	\frac{DZ}{Dt} - \frac{Cg}{f\theta _0}\left(X-x\right) = 0,\\
	P = \frac{1}{2}x^2 + f^{-2}\varphi,\\
	\nabla P = \left( X, Z\right),\\
	\nabla \cdot \bm{u} = 0.
	\end{align*}
	\item Due to the range of scales in the parameters used we found that a number of points are mapped outside the domain $\Gamma$. In this case, as suggested by M\'{e}rigot \cite{Merigot2017} the initial choice of weights as $\psi_i^0 = \color{red}(-)\color{black}d(Z_i,\Gamma)^2$. Additionally we perturb the distance slightly to ensure the mass/area of each cell within the domain $\Gamma$ is positive \color{red} choice of positive sign due to the convention?? used in defining the Voronoi Cells/Laguerre Cells in the code.\color{black} 
	\item The Damped Newton Algorithm is applied to find the set of weights so that the transport map $T_\psi$ is optimal 
	\item Given the points $\left\lbrace X_i,Z_i\right\rbrace $ and weights $\left\lbrace \psi_i \right\rbrace $ we compute the centroids of the Laguerre cells
	\item Transform back to physical co-ordinates and time-step with Euler's method
	 \end{itemize}
	\section{Queries/ideas}
	\begin{itemize}
		\item Physical interpretation of 
		the "weights" in SG model- are they a just a mechanism for finding points that would have zero weights (ie) the Voronoi diagram that satifies the constraints?
		\item pg 68 - 70 Cullen \cite{Cullen2008} transformed energy
		 \begin{equation}
		 E = \int_ \Gamma f^2 \left(\frac{1}{2}\left(\left( \tilde{x} - X\right)^2 +\left(\tilde{y} - Y\right)^2\right) - \tilde{z}Z\right)\sigma dXdYdZ
		 \end{equation}
	       \item what is f in the code? is it the density $\rho$, to overcome scaling issue include $f^2$ factor in cost/density - not completely sure how to do this but tried by multiplying \textquoteleft$f_0$\textquoteright in code by $f^2$ - convergence is a lot better $\epsilon_g = 5.44 e^{-7}$
                 \comments{Colin}{$f$ is just a constant, so minimising $E$ is the same mathematically as minimising $E/f^2$. This factor will influence convergence criteria, though.}
		 \item scaling to unit mass - improved convergence but 'area of cells' is small - causes numerical instabilities
	\end{itemize}

\newpage
\bibliographystyle{unsrt}
\bibliography{Project_bib}
\end{document}
